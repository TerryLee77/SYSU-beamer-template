% !TEX encoding = UTF-8 Unicode
\documentclass[
10pt,
aspectratio=169,
]{beamer}
\setbeamercovered{transparent=10}
\usetheme[
%  showheader,
%  red,
  purple,
%  gray,
%  graytitle,
  colorblocks,
%  noframetitlerule,
]{Verona}

\usepackage[T1]{fontenc}
\usepackage[utf8]{inputenc}
\usepackage{lipsum}
%%%%%%%%%%%%%%%%%%%%%%%%%%%%%%%
% Mac上使用如下命令声明隶书字体,windows也有相关方式,大家可自行修改
\providecommand{\lishu}{\CJKfamily{zhli}}
%%%%%%%%%%%%%%%%%%%%%%%%%%%%%%%
\usepackage{tikz}
\usetikzlibrary{fadings}
%
%\setbeamertemplate{sections/subsections in toc}[ball]
\usepackage{xeCJK}
\usepackage{listings}
\usepackage{caption}
\usepackage{subcaption}
\usefonttheme{professionalfonts}
\def\mathfamilydefault{\rmdefault}
\usepackage{amsmath}
\usepackage{multirow}
\usepackage{booktabs}
\usepackage{bm}
\usepackage{algorithm2e}
\usepackage{listings} 
\setbeamertemplate{section in toc}{\hspace*{1em}\inserttocsectionnumber.~\inserttocsection\par}
\setbeamertemplate{subsection in toc}{\hspace*{2em}\inserttocsectionnumber.\inserttocsubsectionnumber.~\inserttocsubsection\par}
\setbeamerfont{subsection in toc}{size=\small}
\AtBeginSection[]{%
	\begin{frame}%
		\frametitle{Outline}%
		\textbf{\tableofcontents[currentsection]} %
	\end{frame}%
}

%\AtBeginSubsection[]{%
%	\begin{frame}%
%		\frametitle{Outline}%
%		\textbf{\tableofcontents[currentsection, currentsubsection]} %
%	\end{frame}%
%}

\title{大标题}
\subtitle{副标题}
\author[Terry]{姓名}
\mail{\url{邮箱地址}}
\institute[SUN YAT-SEN UNIVERSITY]{指导老师:\\XX学院\\
中山大学}
\date{\today}
\titlegraphic[width=3cm]{sysu_big_logo}{}


% 配置 listings 包的一些基本样式
\lstset{
	basicstyle=\ttfamily\small,
	keywordstyle=\color{blue},
	stringstyle=\color{red},
	commentstyle=\color{green},
	morecomment=[l][\color{magenta}]{\#},
	frame=single,
	breaklines=true,
  postbreak=\mbox{\textcolor{red}{$\hookrightarrow$}\space},
}
%%%%%%%%%%%%%%%%%%%%%%%%%%%%%%%%
% ----------- 标题页 ------------
%%%%%%%%%%%%%%%%%%%%%%%%%%%%%%%%



\begin{document}

\maketitle

%%% define code
%\defverbatim[colored]\lstI{
%	\begin{lstlisting}[language=C++,basicstyle=\ttfamily,keywordstyle=\color{red}]
%	int main() {
%	// Define variables at the beginning
%	// of the block, as in C:
%	CStash intStash, stringStash;
%	int i;
%	char* cp;
%	ifstream in;
%	string line;
%	[...]
%	\end{lstlisting}
%}
%%%%%%%%%%%%%%%%%%%%%%%%%%%%%%%%
% ----------- FRAME ------------
%%%%%%%%%%%%%%%%%%%%%%%%%%%%%%%%

\section{简介}
\subsection{研究背景}
\begin{frame}[c]{简介}
	\framesubtitle{研究背景}
\begin{block}{研究背景及场景}
\begin{itemize}
	\item ABC
\end{itemize}
\end{block}

\begin{block}{研究现状}
	\begin{itemize}
	\item 123456
	\end{itemize}
\end{block}

%\begin{exampleblock}{Example Block}
%	Content of an example block
%\end{exampleblock}
%
%\begin{alertblock}{Alert block}
%	Content of an alert block
%\end{alertblock}

\end{frame}	

\subsection{研究目标}
\begin{frame}[c]{简介}
	\framesubtitle{研究目标}
	\begin{block}{研究目标}
		\begin{enumerate}
			\item ABC
		\end{enumerate}
	\end{block}
	
	\begin{alertblock}{挑战}
		数据不平衡
	\end{alertblock}
		%\begin{exampleblock}{Example Block}
	%	Content of an example block
	%\end{exampleblock}
	%
	%\begin{alertblock}{Alert block}
	%	Content of an alert block
	%\end{alertblock}
	
\end{frame}	

\section{研究方法}



\begin{frame}{研究方法}
	\framesubtitle{模型}
	\begin{equation*}
		J(\theta) = \mathbb{E}_{\pi_\theta}[G_t] = \sum_{s\in\mathcal{S}} d^\pi (s)V^\pi(s)=\sum_{s\in\mathcal{S}} d^\pi(s)\sum_{a\in\mathcal{A}}\pi_\theta(a|s)Q^\pi(s,a)
	\end{equation*}
	
	\begin{align}
		Q_\mathrm{target}&=r+\gamma Q^\pi(s^\prime, \pi_\theta(s^\prime)+\epsilon)\\
		\epsilon&\sim\mathrm{clip}(\mathcal{N}(0, \sigma), -c, c)\nonumber
	\end{align}
\end{frame}

\begin{frame}{模型}
	\begin{alertblock}{编号多行公式}
		% Taken from Mathmode.tex
		\begin{multline}
			A=\lim_{n\rightarrow\infty}\Delta x\left(a^{2}+\left(a^{2}+2a\Delta x+\left(\Delta x\right)^{2}\right)\right.\label{eq:reset}\\
			+\left(a^{2}+2\cdot2a\Delta x+2^{2}\left(\Delta x\right)^{2}\right)\\
			+\left(a^{2}+2\cdot3a\Delta x+3^{2}\left(\Delta x\right)^{2}\right)\\
			+\ldots\\
			\left.+\left(a^{2}+2\cdot(n-1)a\Delta x+(n-1)^{2}\left(\Delta x\right)^{2}\right)\right)\\
			=\frac{1}{3}\left(b^{3}-a^{3}\right)
		\end{multline}
	\end{alertblock}
\end{frame}

\begin{frame}{算法(pseudocode)}
\begin{algorithm}[H]
	\SetAlgoLined
	\KwResult{写出算法的结果}
	初始化变量\;
	\While{条件}{
		执行一些处理\;
		\If{满足某条件}{
			做一些事情\;
		}
	}
	\caption{示例算法}
\end{algorithm}
\end{frame}

\section{应用与结果分析}
\begin{frame}{模型}
	\begin{itemize}
		\item \LaTeX 广泛用于学术界,期刊会议论文模板
	\end{itemize}
	%% 表格演示
	\begin{table}[h]
		\centering
		\begin{tabular}{c|c}
			Microsoft\textsuperscript{\textregistered}  Word & \LaTeX \\
			\hline
			文字处理工具 & 专业排版软件 \\
			容易上手,简单直观 & 容易上手 \\
			所见即所得 & 所见即所想,所想即所得 \\
			高级功能不易掌握 & 进阶难,但一般用不到 \\
			处理长文档需要丰富经验 & 和短文档处理基本无异 \\
			花费大量时间调格式 & 无需担心格式,专心作者内容 \\
			公式排版差强人意 & 尤其擅长公式排版 \\
			二进制格式,兼容性差 & 文本文件,易读、稳定 \\
			付费商业许可 & 自由免费使用 \\
		\end{tabular}
	\end{table}
\end{frame}

\section{总结与期望}
\begin{frame}{总结与期望}
	\begin{figure}[htbp]
		\centering
		\includegraphics[width=0.45\linewidth]{sysu_big_logo.jpeg}
		\caption{图片实例}
		\label{fig:1}
	\end{figure}
\end{frame}

\section{参考文献}
\begin{frame}[allowframebreaks]{参考文献}
	\bibliographystyle{acm} 
	\nocite{*} 
	\bibliography{reference}
\end{frame}

\beamertemplateshadingbackground{structure.fg!90}{structure.fg}
\begin{frame}[plain]
	\vfill
	\centering
	{
		\centering \Huge \color{white} Thank you for your attention!\\[10pt]Q\&A
	}
	\vfill
\end{frame}




\end{document}


